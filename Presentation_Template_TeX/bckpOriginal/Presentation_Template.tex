\documentclass[11pt,xcolor={dvipsnames}]{beamer} % presentation output
% \documentclass[11pt,xcolor={dvipsnames},handout]{beamer} % Beamer printout
% xcolor allows to use many new colors with \usecolortheme

\mode<presentation>{
  \usetheme{Warsaw}  
%  Here is a gallery with other themes:
%  http://deic.uab.es/~iblanes/beamer_gallery/
  \usecolortheme[named=OliveGreen]{structure}
%  Others: OliveGreen, Brown, Sepia, RawSienna, 
  \useoutertheme{shadow}
 	\setbeamercovered{transparent}
	\setbeamercolor{block title example}{fg=white,bg=Blue}
	\setbeamercolor{block body example}{fg=black,bg=Blue!10}
	\setbeamercolor{postit}{fg=black,bg=OliveGreen!20}
	\setbeamercolor{postit2}{fg=yellow,bg=OliveGreen}
%    \setbeamercolor{NEW_STYLE_NAME}{fg=COLOR_FOREGROUNG,bg=COLOR_BACKGROUNG}
}

%% Setting for Beamer printout
% reference: http://mathoverflow.net/questions/5893/beamer-printout
\usepackage{pgfpages}
\mode<handout>{
  \usetheme{default}
  \setbeamercolor{background canvas}{bg=Black!5}
  \pgfpagesuselayout{4 on 1}[a4paper,portrait,border shrink=2.5mm]
  % 4 slide in one page
}
%% Setting for Beamer printout

\usepackage[italian]{babel}
\usepackage[latin1]{inputenc}
\usepackage{times}
\usepackage[T1]{fontenc}
\usepackage{graphics}
\graphicspath{{images/}}
% all the graphics files will go in the subdirectory images
\usepackage{numprint}
% with this one \np{1000} becomes 1 000
\usepackage{mathcomp}
\usepackage{gensymb}
% with this one \numprint[\textcelsius]{20} becomes 20�C 
\newcommand{\ud}{\mathop{}\ \mathrm{d}}
% with this one \ud{x} becomes dx 
\usepackage{mathtools}
\DeclarePairedDelimiter{\abs}{\lvert}{\rvert}
% to define absolute value (mathtools is required)

\hypersetup{
			pdftitle={TITLE OF YOUR PRESENTATION},
			pdfsubject={UNIVERSITY, DEPARTMENT},
			pdfauthor={NAME SURNAME},
			pdfkeywords={KEY1, KEY2, KEY3, etc.},
			pdfpagemode=FullScreen, % once opened it goes in fullscreen modality
			%citecolor=black,
			%filecolor=black,
			%linkcolor=black,
			%urlcolor=black
}

\usepackage[absolute,overlay]{textpos}
\setlength{\TPHorizModule}{1mm}
\setlength{\TPVertModule}{1mm}

%%%% A NEW COMMAND TO FIX LOGO POSITION (x,y) in mm
\newcommand{\MyLogo}{%
\begin{textblock}{14}(2.0,0.1)
%  \pgfuseimage{logo}
 \includegraphics[height=1.15cm, angle=0]{logo}
\end{textblock}
}
%%%% A NEW COMMAND TO FIX LOGO POSITION (x,y) in mm

%%%%%%%%%%%%%%%%%%%%%%%%%%%%%%%%%%%%%%%%%%%%%%%%%%%%%%%%%%%%%%%%%%%%%%%%%

\title[SHORT TITLE OF YOUR PRESENTATION]{COMPLETE TITLE OF YOUR PRESENTATION}
\author[NAME SURNAME]
{NAME SURNAME}
\institute[INSTITUTE NAME]
{
  COMPLETE NAME OF THE INSTITUTE\\
  OTHER INFORMATION \\
  DEPARTMENT NAME \\[0.5cm]
  SUPERVISOR\\ Prof. Eng. \textbf{NAME SURNAME}\\[0.25cm]
  CORRELATOR\\ Eng. \textbf{NAME SURNAME}\\
  }
\date{December 2, 2012}

%\logo{\includegraphics[height=1.5cm, angle=0]{logo}}
% To have a logo on each page... BAD RESULT!!

%\titlegraphic{\includegraphics[height=1.4cm, angle=0]{logo}}
% To have an imagie on title page

%%%% TO HAVE A TOC ON EVERY SLIDE
%\AtBeginSubsection[]
%{
%  \begin{frame}<beamer>{Sommario}
%    \tableofcontents[currentsection,currentsubsection]
%    \tableofcontents[currentsection]
%    \tableofcontents
%  \end{frame}
%}
%%%% TO HAVE A TOC ON EVERY SLIDE

\begin{document}
\transduration{1}

%%%%%%%%%%%%%%%%%%%%%%%%%%%%%    TITLE    %%%%%%%%%%%%%%%%%%%%%%%%%%%%%%%
\begin{frame}
\transdissolve
\MyLogo
\begin{center}
% \includegraphics[height=1.5cm, angle=0]{unipd}
  \titlepage
\end{center}
\end{frame}

%%%% TOC
\begin{frame}{Contents}
\transboxin
\MyLogo
%\tableofcontents[pausesections,part=1]
  \tableofcontents
\end{frame}


%%%%%%%%%%%%%%%%%%%%%%%%%%%% FIRST SECTION %%%%%%%%%%%%%%%%%%%%%%%%%%%%%%
\section{TITLE OF SECTION 1}

\subsection{TITLE OF SUBSECTION 1.1}
\begin{frame}{TITLE OF FRAME 1}
\transboxin
%\transblindshorizontal
% type of transition effect
\MyLogo
\begin{center}
\includegraphics[width=.3\textwidth]{image}
\begin{block}{Block title}
Description of this block. Description of this block. Description of this block. Description of this block. 
\end{block}
\end{center}
\end{frame}

\begin{frame}{TITLE OF FRAME 2}
%\transblindshorizontal
\MyLogo
\begin{center}
\includegraphics[width=.3\textwidth]{image}
\begin{alertblock}{Block title}
Description of this block. Description of this block. Description of this block. Description of this block. \\
\end{alertblock}
\vspace{0.8cm}
\end{center}
\end{frame}

\begin{frame}{TITLE OF FRAME 3}
%\transglitter
\MyLogo
\begin{columns}
\column{.6\textwidth}
	\includegraphics[width=.6\textwidth]{image} 
\column{.4\textwidth}
\begin{block}{Block title}
Description of this block. Description of this block. Description of this block. Description of this block.
\end{block}
\bigskip \bigskip
\begin{block}{Block title}
Description of this block. Description of this block. Description of this block. Description of this block.
\end{block}
\end{columns}
\end{frame}

\begin{frame}{TITLE OF FRAME 4}
%\transglitter
\MyLogo
\begin{columns}
\column{.6\textwidth}
	\includegraphics<1>[width=.5\textwidth]{image1} 
	\includegraphics<2>[width=.5\textwidth]{image} 
	\includegraphics<3>[width=.5\textwidth]{image3}
\column{.4\textwidth}
	\begin{itemize}
  		\item<1-> Text: \alert{text}
  		\item<2-> Text: \alert{text}
		\item<3-> Text: \alert{text}
  	\end{itemize}
\end{columns}
\end{frame}

\begin{frame}{TITLE OF FRAME 5}
%\transdissolve
\MyLogo
\begin{columns}
\column{.4\textwidth}
	\begin{itemize}
  		\item Text text text
  		\item Text text text
		\item Text text text
		\item Text text text
  	\end{itemize}
\column{.6\textwidth}
	\includegraphics [width=.5\textwidth]{image} 
\end{columns}
\end{frame}

%%%%%%%%%%%%%%%%%%%%%%%%%%%%%%%%%%%%%%%%%%%%%%%%%%%%%%%%%%%%%%%%%%%%%%%%%
\begin{frame}{TITLE OF FRAME 6}
\framesubtitle{SUBTITLE OF FRAME 6}
%\transdissolve
\MyLogo
\begin{columns}
\column{.25\textwidth}
\begin{align*}
 \alert{COP_H} &= \dfrac{\abs{Q_2}}{\abs{L}} \\
 	   &= \dfrac{\abs{Q_2}}{\abs{Q_2}-Q_1} \\
 	   &= \alert{\dfrac{T_2}{T_2-T_1}} \\
\end{align*}
\column{.75\textwidth}
	\includegraphics[width=.5\textwidth]{image1}
\end{columns}
\end{frame}

%%%%%%%%%%%%%%%%%%%%%%%%%%%%%%%%%%%%%%%%%%%%%%%%%%%%%%%%%%%%%%%%%%%%%%%%%
\subsection{TITLE OF SUBSECTION 1.2}
\begin{frame}{SUBTITLE OF FRAME 7}
%\transblindshorizontal
\MyLogo
\begin{enumerate}
\item<1-> Text text text text text text text text text text text text text text text text text text text text text text text text text text;
\item<2-> text text text text text text text text text text text text text text text text text text text text text text text text text;
\item<3-> text text text text text text text text text text text text text text text text text text text text text text text text text text text text text text.
\end{enumerate}
\end{frame}

%%%%%%%%%%%%%%%%%%%%%%%%%%%%%%%%%%%%%%%%%%%%%%%%%%%%%%%%%%%%%%%%%%%%%%%%%
\subsection{TITLE OF SUBSECTION 1.3}
\begin{frame}{SUBTITLE OF FRAME 8}
%\transblindshorizontal
\MyLogo
\begin{center}
\includegraphics[width=.2\textwidth]{image1}
\end{center}
\pause
\begin{beamercolorbox}[shadow=false, rounded=true]{postit2}
\begin{itemize}
\item text text text text text text text text text text
\item text text text text 
\item text text text text text text text text text text text text
\item text text text text
\end{itemize}
\end{beamercolorbox}
\end{frame}

\subsection{TITLE OF SUBSECTION 1.4}
\begin{frame}{SUBTITLE OF FRAME 9}
\transdissolve
\MyLogo
\begin{itemize}
\item<1-> Text text text text text text text text text text text text text text text text text text; 
\item<2-> text text text text text text text text text text text text text text text text text text text text text text text text text;
\item<3-> text text text text text text text text text text.
\end{itemize}
\end{frame}


%%%%%%%%%%%%%%%%%%%%%%%%%%%% SECOND SECTION %%%%%%%%%%%%%%%%%%%%%%%%%%%%%
\section{TITLE OF SECTION 2}

\subsection{TITLE OF SUBSECTION 2.1}
\begin{frame}{SUBTITLE OF FRAME 10}
\transboxin
\MyLogo
\begin{itemize}
\item<1-> text text text text text text text text text text text text text text text text text text text text text text text text
\begin{beamercolorbox}[center, shadow=false, rounded=true]{postit2}
text text text text text text text text \numprint[m]{100} text text text text $\numprint[]{4} \div \numprint[kW_t]{7}$
\end{beamercolorbox}
\item<2->  text text text text text text text text text text text text text text text text text text text text
\item<3->  text text text text text text text text text text text text text text text text text text text text
\end{itemize}
\end{frame}

%
% ...
%


%%%%%%%%%%%%%%%%%%%%%%%%%%%%%% LAST FRAME %%%%%%%%%%%%%%%%%%%%%%%%%%%%%%%
\begin{frame}
\transboxin
\MyLogo
\vspace{1.0cm}
\begin{beamercolorbox}[sep=1.0cm, center, shadow=false, rounded=true]{postit2}
\begin{Huge}Thank you for your attention\end{Huge}
\end{beamercolorbox}
\pause
\end{frame}

\end{document}