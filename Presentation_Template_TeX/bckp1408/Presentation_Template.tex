\documentclass[11pt,xcolor={dvipsnames}]{beamer} % presentation output
% \documentclass[11pt,xcolor={dvipsnames},handout]{beamer} % Beamer printout
% xcolor allows to use many new colors with \usecolortheme

\mode<presentation>{
  \usetheme{Warsaw}  
%  Here is a gallery with other themes:
%  http://deic.uab.es/~iblanes/beamer_gallery/
  \usecolortheme[named=OliveGreen]{structure}
%  Others: OliveGreen, Brown, Sepia, RawSienna, 
%  \useoutertheme{shadow}
\defbeamertemplate*{footline}{shadow theme}
{%
  \leavevmode%
  \hbox{\begin{beamercolorbox}[wd=.5\paperwidth,ht=2.5ex,dp=1.125ex,leftskip=.3cm plus1fil,rightskip=.3cm]{author in head/foot}%
    \usebeamerfont{author in head/foot}\hfill\insertshortauthor
  \end{beamercolorbox}%
  \begin{beamercolorbox}[wd=.5\paperwidth,ht=2.5ex,dp=1.125ex,leftskip=.3cm,rightskip=.3cm plus1fil]{title in head/foot}%
    \usebeamerfont{title in head/foot}\insertshorttitle\hfill\insertframenumber\,/\,\inserttotalframenumber%
  \end{beamercolorbox}}%
  \vskip0pt%
}
 	\setbeamercovered{transparent}
	\setbeamercolor{block title example}{fg=white,bg=Blue}
	\setbeamercolor{block body example}{fg=black,bg=Blue!10}
	\setbeamercolor{postit}{fg=black,bg=OliveGreen!20}
	\setbeamercolor{postit2}{fg=yellow,bg=OliveGreen}
%    \setbeamercolor{NEW_STYLE_NAME}{fg=COLOR_FOREGROUNG,bg=COLOR_BACKGROUNG}
}

%% Setting for Beamer printout
% reference: http://mathoverflow.net/questions/5893/beamer-printout
\usepackage{pgfpages}
\mode<handout>{
  \usetheme{default}
  \setbeamercolor{background canvas}{bg=Black!5}
  \pgfpagesuselayout{4 on 1}[a4paper,portrait,border shrink=2.5mm]
  % 4 slide in one page
}
%% Setting for Beamer printout

\usepackage[italian]{babel}
\usepackage[latin1]{inputenc}
\usepackage{times}
\usepackage{verbatim}
\usepackage[T1]{fontenc}
\usepackage{graphics}
\graphicspath{{images/}}
% all the graphics files will go in the subdirectory images
\usepackage{numprint}
% with this one \np{1000} becomes 1 000
\usepackage{mathcomp}
\usepackage{gensymb}
% with this one \numprint[\textcelsius]{20} becomes 20�C 
\newcommand{\ud}{\mathop{}\ \mathrm{d}}
% with this one \ud{x} becomes dx 
\usepackage{mathtools}
\DeclarePairedDelimiter{\abs}{\lvert}{\rvert}
% to define absolute value (mathtools is required)

\hypersetup{
			pdftitle={Classificazione di firme statiche utilizzando i Hidden Markov Models},
			pdfsubject={UNIVERSITY, DEPARTMENT},
			pdfauthor={Alexandru PRIGOREANU},
			pdfkeywords={firme statiche, Hidden Markov Model, hmm, etc.},
			pdfpagemode=FullScreen, % once opened it goes in fullscreen modality
			%citecolor=black,
			%filecolor=black,
			%linkcolor=black,
			%urlcolor=black
}

\usepackage[absolute,overlay]{textpos}
\setlength{\TPHorizModule}{1mm}
\setlength{\TPVertModule}{1mm}

%%%% A NEW COMMAND TO FIX LOGO POSITION (x,y) in mm
\newcommand{\MyLogo}{%
\begin{textblock}{14}(2.0,0.6)
%  \pgfuseimage{logo}
 \includegraphics[height=1.15cm, angle=0]{logoUnipd.png}
\end{textblock}
}
%%%% A NEW COMMAND TO FIX LOGO POSITION (x,y) in mm

%%%%%%%%%%%%%%%%%%%%%%%%%%%%%%%%%%%%%%%%%%%%%%%%%%%%%%%%%%%%%%%%%%%%%%%%%

\title[HMM-based offline signature verification]{Classificazione firme statiche utilizzando i Hidden Markov Models}
\author[Prigoreanu 1004887]
{Alexandru PRIGOREANU}
\institute[INSTITUTE NAME]
{
  {\LARGE Universit� degli studi di Padova}\\[0.3cm]
  {\Large Dipartimento di Matematica}\\
  {\large Corso di laurea in Informatica}\\[0.3cm]
  Relatore\\[0.25cm] Prof. {\large Tullio VARDANEGA}\\[0.25cm]
  }
\date{Dicembre 12, 2013}

%\logo{\includegraphics[height=1.5cm, angle=0]{logo}}
% To have a logo on each page... BAD RESULT!!

%\titlegraphic{\includegraphics[height=1.4cm, angle=0]{logo}}
% To have an imagie on title page

%%%% TO HAVE A TOC ON EVERY SLIDE
%\AtBeginSubsection[]
%{
%  \begin{frame}<beamer>{Sommario}
%    \tableofcontents[currentsection,currentsubsection]
%    \tableofcontents[currentsection]
%    \tableofcontents
%  \end{frame}
%}
%%%% TO HAVE A TOC ON EVERY SLIDE

\begin{document}
%\transduration{1}

%%%%%%%%%%%%%%%%%%%%%%%%%%%%%    TITLE    %%%%%%%%%%%%%%%%%%%%%%%%%%%%%%%
\begin{frame}
%\transdissolve
\MyLogo
\begin{center}
% \includegraphics[height=1.5cm, angle=0]{unipd}
  \titlepage
\end{center}
\end{frame}

%%%% TOC
\begin{frame}{Contenuti}
%\transboxin
\MyLogo
%\tableofcontents[pausesections,part=1]
  \tableofcontents
\end{frame}


%%%%%%%%%%%%%%%%%%%%%%%%%%%% FIRST SECTION %%%%%%%%%%%%%%%%%%%%%%%%%%%%%%
\section{Classificazione di firme statiche}

\subsection{Panoramica}
\begin{frame}{Obiettivo e difficolt�}
%\transboxin
%\transblindshorizontal
% type of transition effect
\MyLogo
\begin{center}
\includegraphics[width=.3\textwidth]{obiettivo}
\begin{block}{Obiettivo}
Decidere se una firma � autentica/falsa
\end{block}
\begin{block}{Variazioni intrapersonali}
Le firme personali hanno grande variabilit�, dovuta allo stato emotivo dei firmatari, alla posizione di raccolta, ecc?
\end{block}
\begin{block}{Differenze interpersonali}
Le firme di persone diverse possiedono caratteristiche elementari distinte
\end{block}
\end{center}
\end{frame}

\begin{comment}
\begin{frame}{TITLE OF FRAME 2}
%\transblindshorizontal
\MyLogo
\begin{center}
\includegraphics[width=.3\textwidth]{image}
\begin{alertblock}{Block title}
Description of this block. Description of this block. Description of this block. Description of this block. \\
\end{alertblock}
\vspace{0.8cm}
\end{center}
\end{frame}

\begin{frame}{TITLE OF FRAME 3}
%\transglitter
\MyLogo
\begin{columns}
\column{.6\textwidth}
	\includegraphics[width=.6\textwidth]{image} 
\column{.4\textwidth}
\begin{block}{Block title}
Description of this block. Description of this block. Description of this block. Description of this block.
\end{block}
\bigskip \bigskip
\begin{block}{Block title}
Description of this block. Description of this block. Description of this block. Description of this block.
\end{block}
\end{columns}
\end{frame}

\begin{frame}{TITLE OF FRAME 4}
%\transglitter
\MyLogo
\begin{columns}
\column{.6\textwidth}
	\includegraphics<1>[width=.5\textwidth]{image1} 
	\includegraphics<2>[width=.5\textwidth]{image} 
	\includegraphics<3>[width=.5\textwidth]{image3}
\column{.4\textwidth}
	\begin{itemize}
  		\item<1-> Text: \alert{text}
  		\item<2-> Text: \alert{text}
		\item<3-> Text: \alert{text}
  	\end{itemize}
\end{columns}
\end{frame}

\begin{frame}{TITLE OF FRAME 5}
%\transdissolve
\MyLogo
\begin{columns}
\column{.4\textwidth}
	\begin{itemize}
  		\item Text text text
  		\item Text text text
		\item Text text text
		\item Text text text
  	\end{itemize}
\column{.6\textwidth}
	\includegraphics [width=.5\textwidth]{image} 
\end{columns}
\end{frame}

%%%%%%%%%%%%%%%%%%%%%%%%%%%%%%%%%%%%%%%%%%%%%%%%%%%%%%%%%%%%%%%%%%%%%%%%%
\begin{frame}{TITLE OF FRAME 6}
\framesubtitle{SUBTITLE OF FRAME 6}
%\transdissolve
\MyLogo
\begin{columns}
\column{.25\textwidth}
\begin{align*}
 \alert{COP_H} &= \dfrac{\abs{Q_2}}{\abs{L}} \\
 	   &= \dfrac{\abs{Q_2}}{\abs{Q_2}-Q_1} \\
 	   &= \alert{\dfrac{T_2}{T_2-T_1}} \\
\end{align*}
\column{.75\textwidth}
	\includegraphics[width=.5\textwidth]{image1}
\end{columns}
\end{frame}
\end{comment}
%%%%%%%%%%%%%%%%%%%%%%%%%%%%%%%%%%%%%%%%%%%%%%%%%%%%%%%%%%%%%%%%%%%%%%%%%
\subsection{TITLE OF SUBSECTION 1.2}
\begin{frame}{SUBTITLE OF FRAME 7}
%\transblindshorizontal
\MyLogo
\begin{enumerate}
\item<1-> Text text text text text text text text text text text text text text text text text text text text text text text text text text;
\item<2-> text text text text text text text text text text text text text text text text text text text text text text text text text;
\item<3-> text text text text text text text text text text text text text text text text text text text text text text text text text text text text text text.
\end{enumerate}
\end{frame}

%%%%%%%%%%%%%%%%%%%%%%%%%%%%%%%%%%%%%%%%%%%%%%%%%%%%%%%%%%%%%%%%%%%%%%%%%
\subsection{TITLE OF SUBSECTION 1.3}
\begin{frame}{SUBTITLE OF FRAME 8}
%\transblindshorizontal
\MyLogo
\begin{center}
\includegraphics[width=.2\textwidth]{image1}
\end{center}
\pause
\begin{beamercolorbox}[shadow=false, rounded=true]{postit2}
\begin{itemize}
\item text text text text text text text text text text
\item text text text text 
\item text text text text text text text text text text text text
\item text text text text
\end{itemize}
\end{beamercolorbox}
\end{frame}

\subsection{TITLE OF SUBSECTION 1.4}
\begin{frame}{SUBTITLE OF FRAME 9}
\transdissolve
\MyLogo
\begin{itemize}
\item<1-> Text text text text text text text text text text text text text text text text text text; 
\item<2-> text text text text text text text text text text text text text text text text text text text text text text text text text;
\item<3-> text text text text text text text text text text.
\end{itemize}
\end{frame}


%%%%%%%%%%%%%%%%%%%%%%%%%%%% SECOND SECTION %%%%%%%%%%%%%%%%%%%%%%%%%%%%%
\section{TITLE OF SECTION 2}

\subsection{TITLE OF SUBSECTION 2.1}
\begin{frame}{SUBTITLE OF FRAME 10}
\transboxin
\MyLogo
\begin{itemize}
\item<1-> text text text text text text text text text text text text text text text text text text text text text text text text
\begin{beamercolorbox}[center, shadow=false, rounded=true]{postit2}
text text text text text text text text \numprint[m]{100} text text text text $\numprint[]{4} \div \numprint[kW_t]{7}$
\end{beamercolorbox}
\item<2->  text text text text text text text text text text text text text text text text text text text text
\item<3->  text text text text text text text text text text text text text text text text text text text text
\end{itemize}
\end{frame}

%
% ...
%


%%%%%%%%%%%%%%%%%%%%%%%%%%%%%% LAST FRAME %%%%%%%%%%%%%%%%%%%%%%%%%%%%%%%
\begin{frame}
\transboxin
\MyLogo
\vspace{1.0cm}
\begin{beamercolorbox}[sep=1.0cm, center, shadow=false, rounded=true]{postit2}
\begin{Huge}Thank you for your attention\end{Huge}
\end{beamercolorbox}
\pause
\end{frame}

\end{document}