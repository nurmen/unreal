\documentclass[11pt,xcolor={dvipsnames}]{beamer} % presentation output
% \documentclass[11pt,xcolor={dvipsnames},handout]{beamer} % Beamer printout
% xcolor allows to use many new colors with \usecolortheme

\mode<presentation>{
  \usetheme{Warsaw}  
%  Here is a gallery with other themes:
%  http://deic.uab.es/~iblanes/beamer_gallery/
  \usecolortheme[named=OliveGreen]{structure}
%  Others: OliveGreen, Brown, Sepia, RawSienna, 
%  \useoutertheme{shadow}
\defbeamertemplate*{footline}{shadow theme}
{%
  \leavevmode%
  \hbox{\begin{beamercolorbox}[wd=.5\paperwidth,ht=2.5ex,dp=1.125ex,leftskip=.3cm plus1fil,rightskip=.3cm]{author in head/foot}%
    \usebeamerfont{author in head/foot}\hfill\insertshortauthor
  \end{beamercolorbox}%
  \begin{beamercolorbox}[wd=.5\paperwidth,ht=2.5ex,dp=1.125ex,leftskip=.3cm,rightskip=.3cm plus1fil]{title in head/foot}%
    \usebeamerfont{title in head/foot}\insertshorttitle\hfill\insertframenumber\,/\,\inserttotalframenumber%
  \end{beamercolorbox}}%
  \vskip0pt%
}
%%%rende transparente quello che sta per comparire
% 	\setbeamercovered{transparent}
	\setbeamercolor{block title example}{fg=white,bg=Blue}
	\setbeamercolor{block body example}{fg=black,bg=Blue!10}
	\setbeamercolor{postit}{fg=black,bg=OliveGreen!20}
	\setbeamercolor{postit2}{fg=yellow,bg=OliveGreen}
%    \setbeamercolor{NEW_STYLE_NAME}{fg=COLOR_FOREGROUNG,bg=COLOR_BACKGROUNG}
}
\newenvironment{changemargin}[2]{%
  \begin{list}{}{%
    \setlength{\topsep}{0pt}%
    \setlength{\leftmargin}{#1}%
    \setlength{\rightmargin}{#2}%
    \setlength{\listparindent}{\parindent}%
    \setlength{\itemindent}{\parindent}%
    \setlength{\parsep}{\parskip}%
  }%
  \item[]}{\end{list}} 
  \newenvironment<>{varblock}[2][1.0\textwidth]{%
  \setlength{\textwidth}{#1}
  \begin{actionenv}#3%
    \def\insertblocktitle{#2}%
    \par%
    \usebeamertemplate{block begin}}
  {\par%
    \usebeamertemplate{block end}%
  \end{actionenv}}

%% Setting for Beamer printout
% reference: http://mathoverflow.net/questions/5893/beamer-printout
\usepackage{pgfpages}
\mode<handout>{
  \usetheme{default}
  \setbeamercolor{background canvas}{bg=Black!5}
  \pgfpagesuselayout{4 on 1}[a4paper,portrait,border shrink=2.5mm]
  % 4 slide in one page
}
%% Setting for Beamer printout

\usepackage[italian]{babel}
\usepackage[latin1]{inputenc}
\usepackage{times}
\usepackage{verbatim}
\usepackage[T1]{fontenc}
\usepackage{graphics}
\graphicspath{{images/}}
% all the graphics files will go in the subdirectory images
\usepackage{numprint}
% with this one \np{1000} becomes 1 000
\usepackage{mathcomp}
\usepackage{gensymb}
% with this one \numprint[\textcelsius]{20} becomes 20�C 
\newcommand{\ud}{\mathop{}\ \mathrm{d}}
% with this one \ud{x} becomes dx 
\usepackage{mathtools}
\DeclarePairedDelimiter{\abs}{\lvert}{\rvert}
% to define absolute value (mathtools is required)

\hypersetup{
			pdftitle={Classificazione di firme statiche utilizzando i Hidden Markov Models},
			pdfsubject={UNIVERSITY, DEPARTMENT},
			pdfauthor={Alexandru PRIGOREANU},
			pdfkeywords={firme statiche, Hidden Markov Model, hmm, etc.},
			pdfpagemode=FullScreen, % once opened it goes in fullscreen modality
			%citecolor=black,
			%filecolor=black,
			%linkcolor=black,
			%urlcolor=black
}

\usepackage[absolute,overlay]{textpos}
\setlength{\TPHorizModule}{1mm}
\setlength{\TPVertModule}{1mm}

%%%% A NEW COMMAND TO FIX LOGO POSITION (x,y) in mm
\newcommand{\MyLogo}{%
\begin{textblock}{14}(2.0,0.6)
%  \pgfuseimage{logo}
 \includegraphics[height=1.15cm, angle=0]{logoUnipd.png}
\end{textblock}
}
%%%% A NEW COMMAND TO FIX LOGO POSITION (x,y) in mm

%%%%%%%%%%%%%%%%%%%%%%%%%%%%%%%%%%%%%%%%%%%%%%%%%%%%%%%%%%%%%%%%%%%%%%%%%

\title[HMM-based offline signature verification]{Classificazione firme statiche utilizzando i Hidden Markov Models}
\author[Prigoreanu 1004887]
{Alexandru PRIGOREANU}
\institute[INSTITUTE NAME]
{
  {\LARGE Universit� degli studi di Padova}\\[0.3cm]
  {\Large Dipartimento di Matematica}\\
  {\large Corso di laurea in Informatica}\\[0.3cm]
  Relatore\\[0.25cm] Prof. {\large Tullio VARDANEGA}\\[0.25cm]
  }
\date{Dicembre 13, 2013}

%\logo{\includegraphics[height=1.5cm, angle=0]{logo}}
% To have a logo on each page... BAD RESULT!!

%\titlegraphic{\includegraphics[height=1.4cm, angle=0]{logo}}
% To have an imagie on title page

%%%% TO HAVE A TOC ON EVERY SLIDE
%\AtBeginSubsection[]
%{
%  \begin{frame}<beamer>{Sommario}
%    \tableofcontents[currentsection,currentsubsection]
%    \tableofcontents[currentsection]
%    \tableofcontents
%  \end{frame}
%}
%%%% TO HAVE A TOC ON EVERY SLIDE

\begin{document}
%\transduration{1}
%%%%%%%%per queste prime 2 slide, nessun numero slide%%%%%%%%%%%%%%%%%%%
\bgroup
\makeatletter
\setbeamertemplate{footline}
{%
  \leavevmode%
  \hbox{\begin{beamercolorbox}[wd=.5\paperwidth,ht=2.5ex,dp=1.125ex,leftskip=.3cm plus1fil,rightskip=.3cm]{author in head/foot}%
    \usebeamerfont{author in head/foot}\hfill\insertshortauthor
  \end{beamercolorbox}%
  \begin{beamercolorbox}[wd=.5\paperwidth,ht=2.5ex,dp=1.125ex,leftskip=.3cm,rightskip=.3cm plus1fil]{title in head/foot}%
    \usebeamerfont{title in head/foot}\insertshorttitle\hfill%
  \end{beamercolorbox}}%
  \vskip0pt%
}
\makeatother
%%%%%%%%%%%%%%%%%%%%%%%%%%%%%    TITLE    %%%%%%%%%%%%%%%%%%%%%%%%%%%%%%%
\begin{frame}
%\transdissolve
\MyLogo
\begin{center}
% \includegraphics[height=1.5cm, angle=0]{unipd}
  \titlepage
\end{center}
\end{frame}

%%%% TOC
\begin{frame}{Contenuti}
%\transboxin
\MyLogo
%\tableofcontents[pausesections,part=1]
  \tableofcontents
\end{frame}
\egroup
\setcounter{framenumber}{0}

%%%%%%%%%%%%%%%%%%%%%%%%%%%% FIRST SECTION %%%%%%%%%%%%%%%%%%%%%%%%%%%%%%
\section{Analisi}

\subsection{Classificazione di firme statiche}
\begin{frame}{Classificazione di firme statiche}
\framesubtitle{Obiettivo e difficolt�}
%\transboxin
%\transblindshorizontal
% type of transition effect
\MyLogo
\begin{center}
\begin{itemize}
  		\item<1-> Obiettivo: Decidere se una firma � autentica/falsa
  		\item<2-> Variazioni intrapersonali: Le firme personali possiedono grande variabilit�, dovuta allo stato emotivo dei sottoscrittori, alla posizione di raccolta, ecc...
		\item<3-> Differenze interpersonali: Le firme di persone diverse possiedono caratteristiche elementari distinte
  	\end{itemize}

	\includegraphics<1>[width=0.5\textwidth , height=50pt]{obiettivo.png}
	\includegraphics<2>[width=1.0\textwidth , height=50pt]{intraPers2.png}
	\includegraphics<3>[width=1.0\textwidth , height=50pt]{interPers.png}

\end{center}
\end{frame}
\begin{frame}{Terminologia}
%\transblindshorizontal
\MyLogo
\begin{center}
\uncover<1-7>{\begin{block}{Tipi di falsificazione}
	\begin{itemize}
  		\item<2-> Falsificazioni Casuali
		\item<3-> Falsificazioni Semplici
		\item<4-> Falsificazioni Accurate
	\end{itemize}
\end{block}}
\uncover<1-7>{\begin{columns}
	\column{.25\textwidth}
		\includegraphics<1->[width=0.9\textwidth , height=40pt]{genuine.png}
	\column{.25\textwidth}
		\includegraphics<2->[width=0.9\textwidth , height=40pt]{rf.png}
	\column{.25\textwidth}
		\includegraphics<3->[width=0.9\textwidth , height=40pt]{simplef.png}
	\column{.25\textwidth}
		\includegraphics<4->[width=0.9\textwidth , height=40pt]{skilledf.png}
\end{columns}}
\uncover<5-7>{\begin{block}{Valutazione della performance}
\begin{itemize}
	\item<6-> \emph{False Acceptance Rate} (\emph{FAR})
	\item<7-> \emph{False Rejection Rate} (\emph{FRR})
\end{itemize}
\end{block}}
\end{center}
\end{frame}

%%%%%%%%%%%%%%%%%%%%%%%%%%%%%%%%%%%%%%%%%%%%%%%%%%%%%%%%%%%%%%%%%%%%%%%%%
\subsection{Processi generali}
\begin{frame}{Processi generali}
%\transblindshorizontal
\MyLogo
\begin{center}
	\includegraphics<1->[width=0.7\textwidth , height=90pt]{generalProcess.png}
\uncover<1->{\begin{columns}
\column{0.5\textwidth}
\uncover<2->{\begin{block}{Preprocessings}
\begin{itemize}
\item \emph{Cropping}
\item \emph{Resizing}
\item \emph{Binarization}
\item \emph{Thinning}
\end{itemize}
\end{block}}
\column{0.5\textwidth}
\uncover<3->{\begin{block}{Features}
\begin{itemize}
\item \emph{Calibre}
\item \emph{Spacing}
\item \emph{Distribution of pixels}
\item \emph{Slant}
\end{itemize}
\end{block}}
\end{columns}}
\end{center}
\end{frame}

%%%%%%%%%%%%%%%%%%%%%%%%%%%%%%%%%%%%%%%%%%%%%%%%%%%%%%%%%%%%%%%%%%%%%%%%%
\subsection{Metodi di classificazione}
\begin{frame}{Metodi di classificazione}
%\transblindshorizontal
\MyLogo
\begin{center}
	\includegraphics<1>[width=0.9\textwidth]{classificatori.png}
	\includegraphics<2>[width=0.9\textwidth]{classificatori1.png}
	\includegraphics<3>[width=0.9\textwidth]{classificatori2.png}
\end{center}
\end{frame}



\subsection{Casi d'uso}
\begin{frame}{Casi d'uso}
%\transdissolve
\MyLogo
\begin{center}
\vspace{-15pt}\includegraphics<1->[width=0.8\textwidth]{uc1.png}
\end{center}
\vspace{-15pt}\uncover<2->{\begin{exampleblock}{Requisito di qualit�}
\begin{itemize}
\item Garantire un'\emph{accuracy} media del 80\%
\end{itemize}
\end{exampleblock}}
\end{frame}


%%%%%%%%%%%%%%%%%%%%%%%%%%%% SECOND SECTION %%%%%%%%%%%%%%%%%%%%%%%%%%%%%
\section{Progettazione}
\subsection{Scelte effettuate}
\begin{frame}{Scelte effettuate}
%\transboxin
\MyLogo
\begin{center}
\vspace{-20pt}
\begin{columns}
\column{0.5\textwidth}
\uncover<1->{\begin{block}{Modello di ciclo di vita}
		\begin{itemize}
			\item\emph{Modello incrementale}
		\end{itemize}
	\end{block}}
	\vspace{5pt}
\uncover<3->{\begin{block}{Librerie Java}
		\begin{itemize}
			\item \emph{ImageJ}
			\item \emph{Jahmm}
			\item \emph{JScience}
			\item \emph{jhmm}
		\end{itemize}
	\end{block}}
\column{0.5\textwidth}
\uncover<2->{\begin{block}{Strumenti}
		\begin{itemize}
			\item \emph{Java}
			\item \emph{Eclipse}
			\item \emph{Hidden Markov Models}
		\end{itemize}
	\end{block}}
		\vspace{5pt}
\uncover<4->{\begin{block}{Design Pattern}
		\begin{itemize}
			\item \emph{Model View Controller}
			\item \emph{Composite}
		\end{itemize}
	\end{block}}
\end{columns}
\end{center}
\end{frame}
%%%%%%%%%%%%%%%%%%%%%%%%%%%%%%%%%%%%%%%%%%%%%%%%%%%%%%%%%%%%%%%%%%%%%%%%
\subsection{Hidden Markov Models}
\begin{frame}{Hidden Markov Models}
\framesubtitle{Definizione}
%\transdissolve
\MyLogo
\begin{center}
\vspace{-10pt}
\begin{block}{${\lambda}=(A,B,{\pi})$}
\begin{itemize}
\item<2-> un insieme S = \{S\ped{1},S\ped{2},...,S\ped{N}\} di stati nascosti
\item<3-> una matrice A = \{a\ped{ij}\}\ :\ a\ped{ij} = P(q\ped{t+1} = S\ped{j}\ |\ q\ped{t} = S\ped{i})
\item<4-> una matrice ${\Pi}$ = \{${\pi}$\ped{i}\}: ${\pi}$\ped{i} = P(q\ped{1} = S\ped{i})\ %con\ 1${\le}$ i${\le}$ N,\  ${\pi}$\ped{i}${\ge}$ 0\  e\  ${\sum\limits_{i=1}^N \pi}$\ped{i} = 1
\item<5-> un insieme V = \{V\ped{1},V\ped{2},...,V\ped{M}\} di simboli di osservazione
\item<6-> al t-esimo istante il processo emette uno fra i simboli a disposizione: o\ped{t} ${\in}$\ \{V\ped{1},V\ped{2},...,V\ped{M}\}
\item<7-> una matrice B = \{b\ped{j}(k)\}\ : b\ped{j}(k) = P(o\ped{t} = k\ |\ q\ped{t} = j)
\item<8-> vale la \emph{propriet� di Markov}:\\
\hspace{-15pt}P(q\ped{t+1} = S\ped{j}\ |\ q\ped{t} = S\ped{i},\ q\ped{t-1} = S\ped{k},\ ...,\ q\ped{1} = S\ped{1})\ =\ P(q\ped{t+1} = S\ped{j}\ |\ q\ped{t} = S\ped{i})
\end{itemize}
\end{block}
\end{center}
\end{frame}

%%%%%%%%%%%%%%%%%%%%%%%%%%%%%%%%%%%%%%%%%%%%%%%%%%%%%%%%%%%%%%%%%%%%%%%%
\begin{frame}{Utilizzo dei HMM}
%\transdissolve
\MyLogo
\begin{center}
\vspace{-10pt}
\begin{block}{Tre problemi (in genere intrattabili)}
\begin{enumerate}
\item<1-> \emph{Evaluation problem}
\item<2-> \emph{Decoding problem}
\item<3-> \emph{Learning problem}
\end{enumerate}
\end{block}
\uncover<4->{\begin{exampleblock}{Tre soluzioni (utilizzando la Programmazione Dinamica)}
\begin{enumerate}
\item<4-> \emph{Forward algorithm}
\item<5-> \emph{Viterbi's algorithm}
\item<6-> \emph{Baum-Welch algorithm}
\end{enumerate}
\end{exampleblock}}
\end{center}
\end{frame}

\begin{frame}{Vettori di osservazione}
%\transdissolve
\MyLogo
%\begin{center}
\vspace{-10pt}
	\begin{columns}
		\hspace{10pt}\column{0.35\textwidth}
			\uncover<1->{\begin{block}{Preprocessings}
				\begin{itemize}
					\item Cropping
					\item Binarization
					\item Skeletonization
					\item Segmentation
				\end{itemize}
			\end{block}}
			\uncover<4->{\begin{block}{Features}
				\begin{itemize}
					\item Slant
					\item DCT
				\end{itemize}
			\end{block}}
		\column{0.9\textwidth}
			\hspace{10pt}\includegraphics<2>[width=0.8\textwidth]{matriceB1.png}
						\includegraphics<3,4>[width=0.8\textwidth]{matriceB2.png}
						\includegraphics<5->[width=0.8\textwidth]{matriceB.png}
	\end{columns}
%\end{center}
\end{frame}

\begin{frame}{Inizializzazione HMM}
%\transdissolve
\MyLogo
\begin{center}
	\includegraphics<1>[width=0.9\textwidth]{training.png}
\end{center}
\end{frame}

%%%%%%%%%%%%%%%%%%%%%%%%%%%% THIRD SECTION %%%%%%%%%%%%%%%%%%%%%%%%%%%%%
\section{Implementazione}
%\subsection{sasasd}
\begin{frame}
%\transboxin
\MyLogo
\begin{center}
\small{
		\hspace{-20pt}\begin{table}[width=0.5\textwidth]
				\hspace{-20pt}\begin{tabular}[width=0.5\textwidth]{l|cccc}
		\hspace{-20pt}Metrica Software & HmmGui & HmmTraining & HmmTesting \\ \hline
		\hspace{-20pt}Complessit� ciclomatica & 2.50 & 5.50 & 2.33 \pause\\
		\hspace{-20pt}Media parametri per metodo & 0.77 & 0.60 & 0.65 \pause\\
		\hspace{-20pt}Linee di codice (\emph{LOC}) per metodo & 13.69 & 26.83 & 12.43 \pause\\
		\hspace{-20pt}Linee di commento per \emph{LOC} & 15.60 & 29.40 & 17.50
		\end{tabular}
\end{table}}
\begin{block}{Verifica e Validazione}
\begin{itemize}
\item Analisi
\item Progettazione
\item Implementazione
\item Documentazione
\end{itemize}
\end{block}
\end{center}
\end{frame}
%%%%%%%%%%%%%%%%%%%%%%%%%%%% FOURTH SECTION %%%%%%%%%%%%%%%%%%%%%%%%%%%%%
\section{Consuntivo}
\subsection{Risultati}
\begin{frame}
%\transboxin
\MyLogo
\begin{center}
\begin{columns}
\column{0.5\textwidth}
	\hspace{10pt}\includegraphics<1->[width=0.75\textwidth]{bestHMM.png}
\column{0.5\textwidth}
\vspace{-10pt}\uncover<2->{\begin{alertblock}{Cause}
\begin{itemize}
\item Scelte progettuali
\item Overfitting
\item Tempo
\end{itemize}
\end{alertblock}}
\uncover<3->{\begin{exampleblock}{Possibili sviluppi}
\begin{itemize}
\item Features
\item Continuous HMM
\item HMM scoring
\item Pseudocounts
\item Regularization
\item Weighting
\item Genetic Algorithms
\end{itemize}
\end{exampleblock}}
\end{columns}
\end{center}
\end{frame}

\subsection{Consuntivo}
\begin{frame}
%\transboxin
\MyLogo
\includegraphics<1->[width=0.7\textwidth]{consuntivo3.png}
\begin{block}{Altre informazioni quantitative}
\begin{itemize}
\item
\end{itemize}
\end{block}
\end{frame}

%%%%%%%%per quest'ultima slide  nessun numero slide%%%%%%%%%%%%%%%%%%%
\bgroup
\makeatletter
\setbeamertemplate{footline}
{%
  \leavevmode%
  \hbox{\begin{beamercolorbox}[wd=.5\paperwidth,ht=2.5ex,dp=1.125ex,leftskip=.3cm plus1fil,rightskip=.3cm]{author in head/foot}%
    \usebeamerfont{author in head/foot}\hfill\insertshortauthor
  \end{beamercolorbox}%
  \begin{beamercolorbox}[wd=.5\paperwidth,ht=2.5ex,dp=1.125ex,leftskip=.3cm,rightskip=.3cm plus1fil]{title in head/foot}%
    \usebeamerfont{title in head/foot}\insertshorttitle\hfill%
  \end{beamercolorbox}}%
  \vskip0pt%
}
\makeatother
\setcounter{framenumber}{12}
%%%%%%%%%%%%%%%%%%%%%%%%%%%%%% LAST FRAME %%%%%%%%%%%%%%%%%%%%%%%%%%%%%%%
\begin{frame}[t]{Ringraziamenti}
\transduration{3}%seconds
\section*{}
\MyLogo
\vspace{-10pt}
\begin{changemargin}{-0.5cm}{-0.5cm}
\'{E} stato un privilegio poter trascorrere questi anni nelle aule e nei laboratori del \emph{Dipartimento di Matematica} dell'\emph{Universit� di Padova} a studiare materie che mi appassionano. Ora che sono giunto quasi alla fine di questo percorso di laurea triennale in \emph{Informatica}, voglio ringraziare le persone il cui contributo non deve essere dimenticato.
\vfill
%\begin{itemize}
\begin{columns}
\column{1.1\textwidth}
\begin{varblock}{}
\only<1-2> {Innanzitutto ringrazio il professor \emph{Vardanega} per avermi seguito durante lo \emph{stage} e per avermi fornito utili consigli per la stesura della relazione.}
\only<3> {In secondo luogo ringrazio i professori del \emph{Corso di Laurea in Informatica} dai quali ho avuto l'opportunit� di ampliare le mie (modeste) conoscenze. Ringrazio i seguenti professori e professoresse (in ordine di apparizione): \emph{Alessandro Sperduti, Federico Menegazzo, Gilberto Fil�, Antonio Grioli, C.E. Palazzi, Carla De Francesco, Lorenzo Finesso, Francesca Rossi, Livio Colussi, Massimo Marchiori, Francesco Tapparo, Paolo Baldan, Silvia Crafa, Luigi De Giovanni, Ombretta Gaggi}.}
\only<4> {Ringrazio l'azienda \emph{Corvallis} e in particolare il \emph{tutor} interno \emph{Alberto Pietrogrande} per l'esperienza di stage vissuta.}
\only<5> {Ringrazio i compagni di corso (ormai amici), coi quali ho svolto alcuni dei progetti didattici, per avermi accettato e sopportato pazientemente (soprattutto quando ero in errore). In particolare ringrazio il gruppo \emph{SevenTech}, formatosi per il progetto di \emph{Ingegneria del Software}, composto da Alessio, Enrico Bo., Enrico Br., Giulio, Nicola, Pietro. Ringrazio il gruppo \emph{IMangiaLibri}, formatosi per il progetto di \emph{Tecnologie Web}, composto da Alberto e Dario. Ringrazio il gruppo \emph{GeoNav}, formatosi per il progetto di \emph{Gestione di Imprese Informatiche}, composto da Fabio e Fabrizio.}
\only<6> {Come ultimo ma non meno importante, ringrazio la mia famiglia per aver reso possibile il proseguimento dei miei studi.}
\end{varblock}
\end{columns}
\end{changemargin}
%\end{itemize}
\vfill
%\pause
\end{frame}
\egroup
\end{document}