\documentclass[11pt,xcolor={dvipsnames}]{beamer} % presentation output
% \documentclass[11pt,xcolor={dvipsnames},handout]{beamer} % Beamer printout
% xcolor allows to use many new colors with \usecolortheme

\mode<presentation>{
  \usetheme{Warsaw}  
%  Here is a gallery with other themes:
%  http://deic.uab.es/~iblanes/beamer_gallery/
  \usecolortheme[named=OliveGreen]{structure}
%  Others: OliveGreen, Brown, Sepia, RawSienna, 
%  \useoutertheme{shadow}
\defbeamertemplate*{footline}{shadow theme}
{%
  \leavevmode%
  \hbox{\begin{beamercolorbox}[wd=.5\paperwidth,ht=2.5ex,dp=1.125ex,leftskip=.3cm plus1fil,rightskip=.3cm]{author in head/foot}%
    \usebeamerfont{author in head/foot}\hfill\insertshortauthor
  \end{beamercolorbox}%
  \begin{beamercolorbox}[wd=.5\paperwidth,ht=2.5ex,dp=1.125ex,leftskip=.3cm,rightskip=.3cm plus1fil]{title in head/foot}%
    \usebeamerfont{title in head/foot}\insertshorttitle\hfill\insertframenumber\,/\,\inserttotalframenumber%
  \end{beamercolorbox}}%
  \vskip0pt%
}
%%%rende transparente quello che sta per comparire
% 	\setbeamercovered{transparent}
	\setbeamercolor{block title example}{fg=white,bg=Blue}
	\setbeamercolor{block body example}{fg=black,bg=Blue!10}
	\setbeamercolor{postit}{fg=black,bg=OliveGreen!20}
	\setbeamercolor{postit2}{fg=yellow,bg=OliveGreen}
%    \setbeamercolor{NEW_STYLE_NAME}{fg=COLOR_FOREGROUNG,bg=COLOR_BACKGROUNG}
}

%% Setting for Beamer printout
% reference: http://mathoverflow.net/questions/5893/beamer-printout
\usepackage{pgfpages}
\mode<handout>{
  \usetheme{default}
  \setbeamercolor{background canvas}{bg=Black!5}
  \pgfpagesuselayout{4 on 1}[a4paper,portrait,border shrink=2.5mm]
  % 4 slide in one page
}
%% Setting for Beamer printout

\usepackage[italian]{babel}
\usepackage[latin1]{inputenc}
\usepackage{times}
\usepackage{verbatim}
\usepackage[T1]{fontenc}
\usepackage{graphics}
\graphicspath{{images/}}
% all the graphics files will go in the subdirectory images
\usepackage{numprint}
% with this one \np{1000} becomes 1 000
\usepackage{mathcomp}
\usepackage{gensymb}
% with this one \numprint[\textcelsius]{20} becomes 20�C 
\newcommand{\ud}{\mathop{}\ \mathrm{d}}
% with this one \ud{x} becomes dx 
\usepackage{mathtools}
\DeclarePairedDelimiter{\abs}{\lvert}{\rvert}
% to define absolute value (mathtools is required)

\hypersetup{
			pdftitle={Classificazione di firme statiche utilizzando i Hidden Markov Models},
			pdfsubject={UNIVERSITY, DEPARTMENT},
			pdfauthor={Alexandru PRIGOREANU},
			pdfkeywords={firme statiche, Hidden Markov Model, hmm, etc.},
			pdfpagemode=FullScreen, % once opened it goes in fullscreen modality
			%citecolor=black,
			%filecolor=black,
			%linkcolor=black,
			%urlcolor=black
}

\usepackage[absolute,overlay]{textpos}
\setlength{\TPHorizModule}{1mm}
\setlength{\TPVertModule}{1mm}

%%%% A NEW COMMAND TO FIX LOGO POSITION (x,y) in mm
\newcommand{\MyLogo}{%
\begin{textblock}{14}(2.0,0.6)
%  \pgfuseimage{logo}
 \includegraphics[height=1.15cm, angle=0]{logoUnipd.png}
\end{textblock}
}
%%%% A NEW COMMAND TO FIX LOGO POSITION (x,y) in mm

%%%%%%%%%%%%%%%%%%%%%%%%%%%%%%%%%%%%%%%%%%%%%%%%%%%%%%%%%%%%%%%%%%%%%%%%%

\title[HMM-based offline signature verification]{Classificazione firme statiche utilizzando i Hidden Markov Models}
\author[Prigoreanu 1004887]
{Alexandru PRIGOREANU}
\institute[INSTITUTE NAME]
{
  {\LARGE Universit� degli studi di Padova}\\[0.3cm]
  {\Large Dipartimento di Matematica}\\
  {\large Corso di laurea in Informatica}\\[0.3cm]
  Relatore\\[0.25cm] Prof. {\large Tullio VARDANEGA}\\[0.25cm]
  }
\date{Dicembre 13, 2013}

%\logo{\includegraphics[height=1.5cm, angle=0]{logo}}
% To have a logo on each page... BAD RESULT!!

%\titlegraphic{\includegraphics[height=1.4cm, angle=0]{logo}}
% To have an imagie on title page

%%%% TO HAVE A TOC ON EVERY SLIDE
%\AtBeginSubsection[]
%{
%  \begin{frame}<beamer>{Sommario}
%    \tableofcontents[currentsection,currentsubsection]
%    \tableofcontents[currentsection]
%    \tableofcontents
%  \end{frame}
%}
%%%% TO HAVE A TOC ON EVERY SLIDE

\begin{document}
%\transduration{1}

%%%%%%%%%%%%%%%%%%%%%%%%%%%%%    TITLE    %%%%%%%%%%%%%%%%%%%%%%%%%%%%%%%
\begin{frame}
%\transdissolve
\MyLogo
\begin{center}
% \includegraphics[height=1.5cm, angle=0]{unipd}
  \titlepage
\end{center}
\end{frame}

%%%% TOC
\begin{frame}{Contenuti}
%\transboxin
\MyLogo
%\tableofcontents[pausesections,part=1]
  \tableofcontents
\end{frame}


%%%%%%%%%%%%%%%%%%%%%%%%%%%% FIRST SECTION %%%%%%%%%%%%%%%%%%%%%%%%%%%%%%
\section{Analisi}

\subsection{Classificazione di firme statiche}
\begin{frame}{Classificazione di firme statiche}
\framesubtitle{Obiettivo e difficolt�}
%\transboxin
%\transblindshorizontal
% type of transition effect
\MyLogo
\begin{center}
\begin{itemize}
  		\item<1-> Obiettivo: Decidere se una firma � autentica/falsa
  		\item<2-> Variazioni intrapersonali: Le firme personali possiedono grande variabilit�, dovuta allo stato emotivo dei sottoscrittori, alla posizione di raccolta, ecc...
		\item<3-> Differenze interpersonali: Le firme di persone diverse possiedono caratteristiche elementari distinte
  	\end{itemize}

	\includegraphics<1>[width=0.5\textwidth , height=50pt]{obiettivo.png}
	\includegraphics<2>[width=1.0\textwidth , height=50pt]{intraPers2.png}
	\includegraphics<3>[width=1.0\textwidth , height=50pt]{interPers.png}

\end{center}
\end{frame}
\begin{frame}{Terminologia}
%\transblindshorizontal
\MyLogo
\begin{center}
\uncover<1-7>{\begin{block}{Tipi di falsificazione}
	\begin{itemize}
  		\item<2-> Falsificazioni Casuali
		\item<3-> Falsificazioni Semplici
		\item<4-> Falsificazioni Accurate
	\end{itemize}
\end{block}}
\uncover<1-7>{\begin{columns}
	\column{.25\textwidth}
		\includegraphics<1->[width=0.9\textwidth , height=40pt]{genuine.png}
	\column{.25\textwidth}
		\includegraphics<2->[width=0.9\textwidth , height=40pt]{rf.png}
	\column{.25\textwidth}
		\includegraphics<3->[width=0.9\textwidth , height=40pt]{simplef.png}
	\column{.25\textwidth}
		\includegraphics<4->[width=0.9\textwidth , height=40pt]{skilledf.png}
\end{columns}}
\uncover<5-7>{\begin{block}{Valutazione della performance}
\begin{itemize}
	\item<6-> \emph{False Acceptance Rate} (\emph{FAR})
	\item<7-> \emph{False Rejection Rate} (\emph{FRR})
\end{itemize}
\end{block}}
\end{center}
\end{frame}

%%%%%%%%%%%%%%%%%%%%%%%%%%%%%%%%%%%%%%%%%%%%%%%%%%%%%%%%%%%%%%%%%%%%%%%%%
\subsection{Processi generali}
\begin{frame}{Processi generali}
%\transblindshorizontal
\MyLogo
\begin{center}
	\includegraphics<1->[width=0.7\textwidth , height=90pt]{generalProcess.png}
\uncover<1->{\begin{columns}
\column{0.5\textwidth}
\uncover<2->{\begin{block}{Preprocessings}
\begin{itemize}
\item \emph{Cropping}
\item \emph{Resizing}
\item \emph{Binarization}
\item \emph{Thinning}
\end{itemize}
\end{block}}
\column{0.5\textwidth}
\uncover<3->{\begin{block}{Features}
\begin{itemize}
\item \emph{Calibre}
\item \emph{Spacing}
\item \emph{Distribution of pixels}
\item \emph{Slant}
\end{itemize}
\end{block}}
\end{columns}}
\end{center}
\end{frame}

%%%%%%%%%%%%%%%%%%%%%%%%%%%%%%%%%%%%%%%%%%%%%%%%%%%%%%%%%%%%%%%%%%%%%%%%%
\subsection{Metodi di classificazione}
\begin{frame}{Metodi di classificazione}
%\transblindshorizontal
\MyLogo
\begin{center}
	\includegraphics<1>[width=0.9\textwidth]{classificatori.png}
	\includegraphics<2>[width=0.9\textwidth]{classificatori1.png}
	\includegraphics<3>[width=0.9\textwidth]{classificatori2.png}
\end{center}
\end{frame}



\subsection{TITLE OF SUBSECTION 1.4}
\begin{frame}{SUBTITLE OF FRAME 9}
\transdissolve
\MyLogo
\begin{itemize}
\item<1-> Text text text text text text text text text text text text text text text text text text; 
\item<2-> text text text text text text text text text text text text text text text text text text text text text text text text text;
\item<3-> text text text text text text text text text text.
\end{itemize}
\end{frame}


%%%%%%%%%%%%%%%%%%%%%%%%%%%% SECOND SECTION %%%%%%%%%%%%%%%%%%%%%%%%%%%%%
\section{TITLE OF SECTION 2}

\subsection{TITLE OF SUBSECTION 2.1}
\begin{frame}{SUBTITLE OF FRAME 10}
\transboxin
\MyLogo
\begin{itemize}
\item<1-> text text text text text text text text text text text text text text text text text text text text text text text text
\begin{beamercolorbox}[center, shadow=false, rounded=true]{postit2}
text text text text text text text text \numprint[m]{100} text text text text $\numprint[]{4} \div \numprint[kW_t]{7}$
\end{beamercolorbox}
\item<2->  text text text text text text text text text text text text text text text text text text text text
\item<3->  text text text text text text text text text text text text text text text text text text text text
\end{itemize}
\end{frame}

%
% ...
%


%%%%%%%%%%%%%%%%%%%%%%%%%%%%%% LAST FRAME %%%%%%%%%%%%%%%%%%%%%%%%%%%%%%%
\begin{frame}
\transboxin
\MyLogo
\vspace{1.0cm}
\begin{beamercolorbox}[sep=1.0cm, center, shadow=false, rounded=true]{postit2}
\begin{Huge}Thank you for your attention\end{Huge}
\end{beamercolorbox}
\pause
\end{frame}

\end{document}